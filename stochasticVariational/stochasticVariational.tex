\documentclass[11pt,a4paper]{article}

\usepackage{epsfig,latexsym,amsbsy,amssymb,amsmath,color, url, natbib, booktabs, multirow,xspace}
\usepackage{colortbl}
\usepackage{longtable}
\usepackage{natbib}
\usepackage{algorithm,algorithmic}
\oddsidemargin  0.0in
\evensidemargin 0.0in
\textwidth      6.5in
\headheight     0.0in
\topmargin     -1.0in
\textheight=10.0in
\parindent=0in 

% genereal purpose macros here
\newcommand{\marg}{\marginpar}
\newcommand{\commentOut}[1]{} 
\newcommand{\add}[1]{\textcolor{red}{#1}} % To highligh modified text
\newcommand{\red}[1]{{\color{red}{#1}}}
\newcommand{\fix}[1]{\red{(#1)} \marginpar{\red{FIX}}}
\newcommand{\todo}[1]{{\fix{TODO: #1}}}


% abbreviations
\newcommand{\etal}{et al.\xspace}
\newcommand{\ie}{i.e.\xspace}
\newcommand{\eg}{e.g.\xspace}
\newcommand{\Ie}{I.e.\xspace}



% Matrices and vectors 
\newcommand{\mat}[1]{\mathbf{#1}}
\renewcommand{\vec}[1]{ \mathbf{#1} } % math bold
\newcommand{\vecS}[1]{\boldsymbol{ #1 }  } % this for boldsymbols
\newcommand{\vecentry}[2]{{\mathrm #1_{#2}}}
\newcommand{\matentry}[3]{{\mathrm #1_{#2,#3}}}
\newcommand{\matcol}[2]{\mat{#1}_{\cdot,#2}}
\newcommand{\matrow}[2]{\mat{#1}_{#2,\cdot}}

 % Matrices and vectors here
 \newcommand{\A}{\mat{A}}
\newcommand{\B}{\mat{B}}
\newcommand{\C}{\mat{C}}
\newcommand{\D}{\mat{D}}
\newcommand{\E}{\mat{E}}
\newcommand{\F}{\mat{F}}
\newcommand{\G}{\mat{G}}
\newcommand{\I}{\mat{I}}
\newcommand{\K}{\mat{K}}
\renewcommand{\L}{\mat{L}}
\newcommand{\Q}{\mat{Q}}
\newcommand{\X}{\mat{X}}
\newcommand{\W}{\mat{W}}
\newcommand{\Y}{\mat{Y}}
\newcommand{\Z}{\mat{Z}}


 % Vectorial greek letters
 \newcommand{\vecalpha}{\vecS{\alpha}}
 \newcommand{\vecphi}{\vecS{\phi}}
 \newcommand{\veceta}{\vecS{\eta}}
  
 
% caligraphic alphabet
\newcommand{\calA}{\mathcal{A}}
\newcommand{\calC}{\mathcal{C}}
\newcommand{\calD}{\mathcal{D}} 
\newcommand{\calL}{\mathcal{L}}
\newcommand{\calM}{\mathcal{M}}
\newcommand{\calT}{\mathcal{T}}
\newcommand{\calU}{\mathcal{U}}
\newcommand{\calX}{\mathcal{X}}
\newcommand{\calF}{\mathcal{F}}
\newcommand{\calE}{\mathcal{E}}
\newcommand{\calI}{\mathcal{I}}
\newcommand{\calS}{\mathcal{S}}
\newcommand{\calY}{\mathcal{Y}}
\newcommand{\calO}{\mathcal{O}}
\newcommand{\calQ}{\mathcal{Q}}
\newcommand{\calR}{\mathcal{R}}
\newcommand{\calV}{\mathcal{V}}

% Tilde Alphabet
\newcommand{\tildeN}{\tilde{N}}

% blackboard alphabet 
\newcommand{\setR}{\mathbb{R}}
\newcommand{\setE}{\mathbb{E}}
\newcommand{\setV}{\mathbb{V}}
\newcommand{\setI}{\mathbb{I}}
\newcommand{\setX}{\mathbb{X}}
\newcommand{\setT}{\mathbb{T}}
\newcommand{\setU}{\mathbb{U}}



% Useful math operators
\newcommand{\Sum}{{\displaystyle \sum}}
\DeclareMathOperator{\vect}{vec}
\DeclareMathOperator{\diag}{diag}
\DeclareMathOperator*{\argmax}{argmax}                 
\DeclareMathOperator*{\argmin}{argmin}                 
\providecommand{\abs}[1]{\lvert#1\rvert}
\providecommand{\norm}[1]{\lVert#1\rVert}
\newcommand{\notp}[1]{\stackrel{\neg}{#1}} % symbol for not present


% matrix products
\newcommand{\kron}{\otimes}
\newcommand{\hada}{\odot}

% statistics
\newcommand{\expectation}{\mathbb{E}}
\newcommand{\Eb}[1]{\left\langle #1 \right\rangle} % Expectation in angle brackets
\newcommand{\variance}{\mathbb{V}}
\newcommand{\covariance}{\mathbb{C}\text{ov}}
\newcommand{\Normal}{\mathcal{N}}  
\newcommand{\avg}[1]{\overline{#1}}


% GP things
\newcommand{\GP}{\mathcal{GP}}
\newcommand{\kernel}{\kappa}

% Other maths
\newcommand{\deriv}[2]{\frac{\partial{#1}}{\partial{#2}}}
\newcommand{\fullderiv}[2]{\frac{d{#1}}{d{#2}}}
\newcommand{\secondderiv}[2]{\frac{\partial^{2}{#1}}{\partial{#2^{2}}}}
\newcommand{\derivtwo}[2]{\frac{\partial^{2}{#1}}{\partial{#2}}}
\newcommand{\gradient}{\nabla}
\newcommand{\hessian}{\nabla\nabla}
\newcommand{\hessentry}[2]{\frac{\partial^{2}}{\partial{#1}\partial{#2}}}
\newcommand{\trace}{\mbox{ \rm tr }}
\renewcommand{\det}[1]{\left\lvert#1\right\rvert}
\newcommand{\defeq}{\stackrel{\text{\tiny def}}{=}}
\newcommand{\idx}{\mathcal{I}}
\newcommand{\T}{\text{T}}
\newcommand{\mth}{\mathrm{th}} 


\newcommand{\bigO}{\calO}



\newcommand{\kl}[2]{\mathrm{KL}(#1 \lVert #2)}


















\newcommand{\vectheta}{\vecS{\theta}}
\newcommand{\param}{\vecS{\lambda}}
\newcommand{\obs}{\vec{y}}
\newcommand{\slike}[1]{\log p(y_{#1}|\vectheta)}
\newcommand{\slikesample}[2]{\log p(y_{#1}|\vectheta^{(#2)})}
\newcommand{\prior}{p(\vectheta)}
\newcommand{\like}[1]{\log p(\obs_{#1} | \vectheta)}
\newcommand{\post}{q(\vectheta | \param)}
\newcommand{\nelbo}{{\mathcal{L}}}
\newcommand{\ellterm}{{{\mathcal{L}}}_{\text{ell}}}
\newcommand{\elltermhat}{{\widehat{\mathcal{L}}}_{\text{ell}}}
\newcommand{\klterm}{{{\mathcal{L}}}_{\text{kl}}}
\newcommand{\singlenelbo}[1]{\nelbo^{(#1)}}
\newcommand{\stepsize}[1]{\alpha_{#1}}
\newcommand{\batch}[1]{{\mathcal{Y}}_{#1}}
\newcommand{\batchsize}[1]{\abs{\batch{#1}}}

\title{Stochastic Variational Inference --- Mini-batch Optimization}
\author{Edwin V.~Bonilla}
\date{Last update: June 22nd, 2017}
\begin{document}
\maketitle
\begin{abstract}
This note clarifies the objective function and its gradients for the case of mini-batch optimization
in stochastic variational inference. This is needed to make the reader aware that the 
KL-divergence term in the variational objective needs to be re-weighted so it matches the 
function decomposition assumed by stochastic optimization algorithms.	This can also be seen from an alternative point of view where
the ELL term is re-weighted yielding a doubly-stochastic estimate. Finally, we analyze this estimate in the case of multi-task learning with  likelihoods that decompose over tasks. 
\end{abstract}

\section{The variational objective}
For a prior $\prior$, an iid likelihood $\like{}$ and a variational posterior 
$\post$ one can usually write the negative evidence lower bound as:
\begin{align}
\nelbo &=  \overbrace{ - \expectation{\post}{\like{}} }^{\ellterm} + \kl{\post}{\prior} \\
&= \sum_{i=1}^{N} - \expectation{\post}{\slike{i}} +  \kl{\post}{\prior}\\
&= \sum_{i=1}^{N}  \underbrace{\left(  - \expectation{\post}{\slike{i}}  + \frac{1}{N}\kl{\post}{\prior} \right)}_{\singlenelbo{i}}
\end{align}
%
\subsection{Optimization of decomposable functions}
By writing the variational objective as a sum of individual objectives we can now apply standard stochastic optimization algorithms such as SGD.
Therefore, if one was to carry out optimization via SGD, the update will be:
\begin{align}
\param_{k+1} &= \param_k - \stepsize{k} \grad_{\param}{\singlenelbo{k}} \\
&= \param_k - \stepsize{k} \left(- \grad_{\param}{\expectation{\post}{\slike{k}}} + \frac{1}{N} \grad_{\param} \kl{\post}{\prior} \right) \text{.}
\end{align}
Now let us assume the more general case of accessing a batch $\batch{b}$ of size $\batchsize{b}$, then
the corresponding noisy gradient estimate is:
\begin{align}
\label{eq:minibatch}
	\grad_{\param}\singlenelbo{b} = - \frac{1}{\batchsize{b}}\sum_{i \in \batch{b}} 
	 \grad_{\param} \expectation{\post}{\slike{i}} + \frac{1}{N}  \grad_{\param} \kl{\post}{\prior} \text{,}
\end{align}
where we see that the relative weighting between the expected log likelihood and the KL term is mantained. 

Equation \eqref{eq:minibatch} is the more general form as it considers batches of different sizes.
If all the batches are of equal size such that $\batchsize{b}=M \ \forall b$, then the number of 
batches is given by $N_B = N/M$.  Then multiplying Equation  \eqref{eq:minibatch} by $M$ we have:
\begin{align}
	\grad_{\param}\singlenelbo{b} &= - \sum_{i \in \batch{b}}  \grad_{\param} \expectation{\post}{\slike{i}} + \frac{1}{N_B}  \grad_{\param}  \kl{\post}{\prior}\text{,} \\
		\grad_{\param}\singlenelbo{b} &= - \sum_{i \in \batch{b}}  \grad_{\param} \expectation{\post}{\slike{i}} + \frac{M}{N} \grad_{\param}  \kl{\post}{\prior}\text{.}
\end{align}
this means that, at each iteration of SGD, we need just to compute the gradients of the expected likelihood (ELL) term for a batch and the weighted KL term.
 We see that, in terms of optimization, an equivalent expression is given by weighting the ELL term instead:
 \begin{align}
 	\label{eq:gradfinal}
 		\grad_{\param}\singlenelbo{b} &= -  \frac{N}{M}  \sum_{i \in \batch{b}}  \grad_{\param} \expectation{\post}{\slike{i}} +  \grad_{\param}  \kl{\post}{\prior}\text{.}
 \end{align} 
%
\section{Estimation of the ELL and doubly-stochastic estimates}
Let us now consider a Monte Carlo (MC) estimate of the individual ELL terms above using samples $S$ from the posterior:
\begin{align}
	\vectheta^{(s)} & \sim \post \text{, } s=1, \ldots, S \\
	\expectationhat{\post}{\slike{i}}  &= \frac{1}{S} \sum_{s=1}^S  \slikesample{i}{s} \text{,}
\end{align}
hence the ELL term is given by:
\begin{equation}
 \elltermhat = 	- \frac{N}{M}  \sum_{i \in \batch{b}}  \frac{1}{S} \sum_{s=1}^S  \slikesample{i}{s} \text{,}
\end{equation}
which is usually known as a doubly-stochastic estimate, as we first estimate the ELL by averaging over a batch and  multiplying by $N$, and then we estimate the expectation using MC.  
%
\section{Multi-task learning with decomposable likelihoods}
Here we analyze the ELL estimate in the case of multi-task learning with $P$ different outputs and a likelihood that decompose over tasks. Let us denote $y_{nt}$ as the label for task $t$ on datapoint $n$, with $n=1,\ldots, N_t$ and  $t=1, \ldots, P$. I general, these observations do not lie on a grid, i.e.~we do not observe the labels for all tasks at the same input. A simple MC estimate using all the data is given by:
\begin{align}
	\elltermhat = \frac{1}{S} \sum_{t=1}^{P} \sum_{n=1}^{N_t} \sum_{s=1}^S \slikesample{nt}{s} \text{.}
\end{align}
The easiest way to handle this case for mini-batch optimization is to  collect all the data into the same pot (while keeping track of the identity of the task) and make  $N= \sum_{t=1}^P N_t$. This will allow us to use SGD as above by selecting a batch of datapoints (in general from several tasks) and apply the same weighting as described above in the single-task case:
\begin{equation}
\elltermhat = 	- \frac{N}{M}  \sum_{i \in \batch{b}}  \frac{1}{S} \sum_{s=1}^S  \slikesample{i}{s} \text{,}
\end{equation}
where the index $i=[n,t]$ now selects a datapoint for a particular task. As above, instead of weighting the ELL we can alternartively weight the KL by $M/N$. 

The above scheme will allow us to raise interest questions. For example, in the active learning scenario, what tasks should I select so I gain the most information about my model? etc.













\end{document}