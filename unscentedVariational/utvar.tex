\documentclass[11pt,a4paper]{article}

\usepackage{epsfig,latexsym,amsbsy,amssymb,amsmath,color, url, natbib, booktabs, multirow,xspace}
\usepackage{colortbl}
\usepackage{longtable}
\usepackage{natbib}
\usepackage{algorithm,algorithmic}
\oddsidemargin  0.0in
\evensidemargin 0.0in
\textwidth      6.5in
\headheight     0.0in
\topmargin     -1.0in
\textheight=10.0in
\parindent=0in 

% genereal purpose macros here
\newcommand{\marg}{\marginpar}
\newcommand{\commentOut}[1]{} 
\newcommand{\add}[1]{\textcolor{red}{#1}} % To highligh modified text
\newcommand{\red}[1]{{\color{red}{#1}}}
\newcommand{\fix}[1]{\red{(#1)} \marginpar{\red{FIX}}}
\newcommand{\todo}[1]{{\fix{TODO: #1}}}


% abbreviations
\newcommand{\etal}{et al.\xspace}
\newcommand{\ie}{i.e.\xspace}
\newcommand{\eg}{e.g.\xspace}
\newcommand{\Ie}{I.e.\xspace}



% Matrices and vectors 
\newcommand{\mat}[1]{\mathbf{#1}}
\renewcommand{\vec}[1]{ \mathbf{#1} } % math bold
\newcommand{\vecS}[1]{\boldsymbol{ #1 }  } % this for boldsymbols
\newcommand{\vecentry}[2]{{\mathrm #1_{#2}}}
\newcommand{\matentry}[3]{{\mathrm #1_{#2,#3}}}
\newcommand{\matcol}[2]{\mat{#1}_{\cdot,#2}}
\newcommand{\matrow}[2]{\mat{#1}_{#2,\cdot}}

 % Matrices and vectors here
 \newcommand{\A}{\mat{A}}
\newcommand{\B}{\mat{B}}
\newcommand{\C}{\mat{C}}
\newcommand{\D}{\mat{D}}
\newcommand{\E}{\mat{E}}
\newcommand{\F}{\mat{F}}
\newcommand{\G}{\mat{G}}
\newcommand{\I}{\mat{I}}
\newcommand{\K}{\mat{K}}
\renewcommand{\L}{\mat{L}}
\newcommand{\Q}{\mat{Q}}
\newcommand{\X}{\mat{X}}
\newcommand{\W}{\mat{W}}
\newcommand{\Y}{\mat{Y}}
\newcommand{\Z}{\mat{Z}}


 % Vectorial greek letters
 \newcommand{\vecalpha}{\vecS{\alpha}}
 \newcommand{\vecphi}{\vecS{\phi}}
 \newcommand{\veceta}{\vecS{\eta}}
  
 
% caligraphic alphabet
\newcommand{\calA}{\mathcal{A}}
\newcommand{\calC}{\mathcal{C}}
\newcommand{\calD}{\mathcal{D}} 
\newcommand{\calL}{\mathcal{L}}
\newcommand{\calM}{\mathcal{M}}
\newcommand{\calT}{\mathcal{T}}
\newcommand{\calU}{\mathcal{U}}
\newcommand{\calX}{\mathcal{X}}
\newcommand{\calF}{\mathcal{F}}
\newcommand{\calE}{\mathcal{E}}
\newcommand{\calI}{\mathcal{I}}
\newcommand{\calS}{\mathcal{S}}
\newcommand{\calY}{\mathcal{Y}}
\newcommand{\calO}{\mathcal{O}}
\newcommand{\calQ}{\mathcal{Q}}
\newcommand{\calR}{\mathcal{R}}
\newcommand{\calV}{\mathcal{V}}

% Tilde Alphabet
\newcommand{\tildeN}{\tilde{N}}

% blackboard alphabet 
\newcommand{\setR}{\mathbb{R}}
\newcommand{\setE}{\mathbb{E}}
\newcommand{\setV}{\mathbb{V}}
\newcommand{\setI}{\mathbb{I}}
\newcommand{\setX}{\mathbb{X}}
\newcommand{\setT}{\mathbb{T}}
\newcommand{\setU}{\mathbb{U}}



% Useful math operators
\newcommand{\Sum}{{\displaystyle \sum}}
\DeclareMathOperator{\vect}{vec}
\DeclareMathOperator{\diag}{diag}
\DeclareMathOperator*{\argmax}{argmax}                 
\DeclareMathOperator*{\argmin}{argmin}                 
\providecommand{\abs}[1]{\lvert#1\rvert}
\providecommand{\norm}[1]{\lVert#1\rVert}
\newcommand{\notp}[1]{\stackrel{\neg}{#1}} % symbol for not present


% matrix products
\newcommand{\kron}{\otimes}
\newcommand{\hada}{\odot}

% statistics
\newcommand{\expectation}{\mathbb{E}}
\newcommand{\Eb}[1]{\left\langle #1 \right\rangle} % Expectation in angle brackets
\newcommand{\variance}{\mathbb{V}}
\newcommand{\covariance}{\mathbb{C}\text{ov}}
\newcommand{\Normal}{\mathcal{N}}  
\newcommand{\avg}[1]{\overline{#1}}


% GP things
\newcommand{\GP}{\mathcal{GP}}
\newcommand{\kernel}{\kappa}

% Other maths
\newcommand{\deriv}[2]{\frac{\partial{#1}}{\partial{#2}}}
\newcommand{\fullderiv}[2]{\frac{d{#1}}{d{#2}}}
\newcommand{\secondderiv}[2]{\frac{\partial^{2}{#1}}{\partial{#2^{2}}}}
\newcommand{\derivtwo}[2]{\frac{\partial^{2}{#1}}{\partial{#2}}}
\newcommand{\gradient}{\nabla}
\newcommand{\hessian}{\nabla\nabla}
\newcommand{\hessentry}[2]{\frac{\partial^{2}}{\partial{#1}\partial{#2}}}
\newcommand{\trace}{\mbox{ \rm tr }}
\renewcommand{\det}[1]{\left\lvert#1\right\rvert}
\newcommand{\defeq}{\stackrel{\text{\tiny def}}{=}}
\newcommand{\idx}{\mathcal{I}}
\newcommand{\T}{\text{T}}
\newcommand{\mth}{\mathrm{th}} 


\newcommand{\bigO}{\calO}



\newcommand{\kl}[2]{\mathrm{KL}(#1 \lVert #2)}


















\newcommand{\ut}{{\sc ut}\xspace}
\newcommand{\x}{\vec{x}}
\newcommand{\y}{\vec{y}}
\renewcommand{\xi}{\x^{(i)}}
\newcommand{\yi}{\y^{(i)}}
%\renewcommand{\A}{\mat{A}}
\renewcommand{\b}{\vec{b}}
\newcommand{\xbar}{\bar{\x}}
\newcommand{\ybar}{\bar{\y}}
\newcommand{\Sigmaxx}{\mat{\Sigma}_{xx}}
\newcommand{\Sigmayx}{\mat{\Sigma}_{yx}}
\newcommand{\grad}{\nabla}
\newcommand{\loss}{\calL(\A,\b)}

\newcommand{\w}{u}
\newcommand{\wi}{u_i}
\newcommand{\wbar}{\bar{\w}}
\newcommand{\wo}{\w_0}


\newcommand{\qx}{q(\x)}
\newcommand{\m}{\vec{m}}
%\newcommand{\C}{\mat{C}}
\newcommand{\xo}{\x^{(0)}}
\renewcommand{\Q}{Q}
\renewcommand{\P}{P}
\renewcommand{\k}{\kappa}
\newcommand{\mathcol}[2]{  \left[#1\right]_{\cdot, #2} } 
\newcommand{\rks}{{\sc rks}}
\newcommand{\ugp}{{\sc ugp}}
\newcommand{\f}{\vec{f}}
\newcommand{\fn}{\f_{n \cdot}}
\newcommand{\An}{\mat{A}_n} 
\newcommand{\bn}{\vec{b}_n} 
\renewcommand{\W}{\mat{W}}
\newcommand{\phin}{\vecS{\phi}_n}
\newcommand{\mufn}{\vecS{\mu}_{fn}}
\newcommand{\Sigmafn}{\mat{\Sigma}_{fn}}
\newcommand{\M}{\mat{M}}
\renewcommand{\D}{D} 
\newcommand{\fnq}{f_{nq}}
\newcommand{\fni}{f_{ni}}
\newcommand{\var}[1]{\mathbb{V}[#1]}
\newcommand{\sigmafn}[1]{\vec{e}_{fn}^{(#1)}}

\title{Using the Unscented Transform in Variational Inference}
\author{Edwin V.~Bonilla}
\begin{document}
\maketitle
We derive the updates for the linearization parameters of a non-linear forward model
when used with a Random Kitchen Sink (\rks) approximation to a multiple output
extension of the unscented Gaussian process of \fix{Steinberg and Bonilla (2014)}. 
The unscented transform (\ut) is useful within a variational inference framework,
when we fix the posterior parameters and want to update the linear model. 
%
We start with deriving the equations for statistical linearization (simply fitting a linear model) 
through least squares, then using weighted least squares (when each observation has a 
weight associated with it) and show the equivalent expressions when the \ut
is used to generate the training data.

\section{Statistical Linearization}
Given $N$ observations $\{ \xi, \yi\}_{i=1}^N$ of a non linear function $\y = g(\x)$, where 
$\x \in \Re^\Q$ and $\y \in \Re^\P$ we want to fit a linear model of the form:
\begin{align}
	f(\x) = \A \x + \b \text{,}
\end{align}
where $\A$ is a $\P \times  \Q$ matrix and $\b$ is a $\P$-dimensional vector, so as to 
minimize the squared loss:
\begin{align}
	\label{eq:loss}
	\loss = \sum_{i=1}^N \norm{\A \xi + \b - \yi }_2^2 \text{.}
\end{align}
Let us define the following statistics:
\begin{align}
	\xbar & = \frac{1}{N} \sum_{i=1}^N \xi \\
	\ybar & = \frac{1}{N} \sum_{i=1}^N \yi \\
	\Sigmaxx & = \sum_{i=1}^N (\xi - \xbar) (\xi - \xbar)^T \\
	\Sigmayx & = \sum_{i=1}^N (\yi - \ybar) (\xi - \xbar)^T \text{.}
\end{align}
Taking the gradient of Equation \eqref{eq:loss} with respect to $\b$  and equating it to zero we have that:
\begin{align}
	\grad_{\b} \loss &= 2 \sum_{i=1}^N ( \A \xi + \b - \yi )   = 0 \\
	\label{eq:b}
		\b & = \ybar - \A \xbar \text{.}
\end{align}
Replacing the value of this in Equation $\eqref{eq:loss}$:
\begin{align}
	\loss &= \sum_{i=1}^N \norm{\A (\xi  -\xbar) -  (\yi - \ybar) }_2^2 \\
	\grad_{\A} \loss & =\sum_{i=1}^N \left(\A (\xi  -\xbar) -  (\yi - \ybar) \right) (\xi  -\xbar)^T = 0 \\	
	& = \A \sum_{i=1}^N (\xi  -\xbar)  (\xi  -\xbar)^T = \sum_{i=1}^N (\yi - \ybar)  (\xi  -\xbar)^T \\
	\A \Sigmaxx & = \Sigmayx \\
		\label{eq:A}
	\A & =  \Sigmayx  \Sigmaxx^{-1} \text{.}
\end{align}
Therefore, the solution to our linearization problem is given by Equation \eqref{eq:b} and
\eqref{eq:A}. 
\section{Weighted Least Squares}
In weighted least squares each observation has a weight $\{ \wi \}$ associated with it (e.g.~an inverse
variance) and the corresponding loss is:
\begin{equation}
	 \loss  = \sum_{i=1}^N \w_i  \norm{\A \xi + \b - \yi }_2^2  \text{.}
\end{equation}
The solution is similar to that obtained before:
\begin{align}
	\label{eq:b-w}
	\b & = \frac{1}{\wbar} (\ybar - \A \xbar) \\
	\label{eq:A-w}	
	\A & = \Sigmayx \Sigmaxx^{-1} \text{,}
\end{align}
where we have (re)defined the statistics:
\begin{align}
	\wbar & = \sum_{i=1}^N \wi\\	
	\xbar & =  \sum_{i=1}^N \wi \xi \\
	\ybar & =  \sum_{i=1}^N \wi \yi \\
	\label{eq:Sigmaxx-w}
	\Sigmaxx & = \sum_{i=1}^N \wi (\xi - \xbar) (\xi - \xbar)^T \\
	\label{eq:Sigmayx-w}
	\Sigmayx & = \sum_{i=1}^N \wi (\yi - \ybar) (\xi - \xbar)^T \text{.}
\end{align}
\section{Variational Inference and the Unscented Transform}
In variational inference we need to compute an expectation of a non-linear 
function $g(\x)$ over the current posterior estimate. When the  
posterior is a $\Q$-dimensional Gaussian $\qx = \Normal(\x; \m, \C)$, we can linearize 
$g(\x) \approx \A \x + \b$ around the current posterior estimate 
and use weighted least squares. The main question 
is what ``training" data  can we use to fit the linear model? Although we can 
sample from $\qx$ to generate these data, the unscented transform (\ut) provides 
a deterministic and more  elegant solution. 

The \ut revolves around the definition of $N=2Q+1$ so-called sigma-points:
\begin{align}
	\xo & = \m \text{,} \\
	\xi &= \m + \mathcol{\sqrt{(\Q + \k) \C}}{i} \quad i = 1, \ldots, \Q \\
	\xi &= \m - \mathcol{\sqrt{(\Q + \k) \C}}{i-\Q} \quad i = \Q+1, \ldots, 2\Q \text{,} \\
% \end{align}
& \text { and the corresponding observations with weights:} \\
% \begin{align}
	\yi & = g(\xi) \quad i = 0, \quad \ldots 2\Q\\
	\wo &= \frac{\k}{\Q + \k} \\
	\wi &= \frac{1}{2(\Q + \k)}  \quad  i = 1, \ldots, 2\Q  \text{,}
\end{align}	
where $\mathcol{\mat{B}}{i}$ denotes the $i\mth$ column of matrix $\mat{B}$.
We note in passing that $\k=1/2$ corresponds to uniform weights $\wi = 1/(2\Q+1)$. \\

Having the training data generated by the \ut, now it is simply a matter of
deriving the expressions for the statistics $\wbar$, $\xbar$, $\ybar$, $\Sigmaxx$, $\Sigmayx$
and the linearization parameters $\A, \b$. 

For the weights we have that:
\begin{equation}
	\wbar = \sum_{i=0}^{2\Q} \wi = \frac{\k}{\Q + \k} + \frac{2\Q}{2 (\Q + \k)} = 1 \text{.}
\end{equation}
Similarly for $\xbar$:
\begin{align}
	\xbar &= \sum_{i=o}^{2\Q} \wi \xi \\
	&= \frac{\k}{\Q + \k} \m + \frac{1}{2(\Q + \k)} \sum_{i=1}^{2\Q} \xi \\
	& = \frac{\k}{\Q + \k} \m + \frac{2 \Q}{2 (\Q + \k)} \m  = \wbar \m  \\
	&= \m.
\end{align}
For the output statistic $\ybar$ is the same as before:
\begin{equation}
	\label{eq:ybar-ut}
	\ybar  =  \sum_{i=0}^{2 \Q} \wi \yi \text{.}
\end{equation}
%
The statistic $\Sigmayx$, using Equation 	\eqref{eq:Sigmayx-w}, is given by:
\begin{align}
	\Sigmayx & = \sum_{i=0}^{2\Q} \wi (\yi - \ybar) (\xi - \xbar)^T \\
	\label{eq:Sigmayx-ut}
	& = \sum_{i=0}^{2\Q} \wi (\yi - \ybar) (\xi - \m)^T \text{.}
\end{align}
Similarly, from Equation \eqref{eq:Sigmaxx-w}, $\Sigmaxx$ is:
\begin{align}
\Sigmaxx & = \sum_{i=0}^{2 \Q} \wi (\xi - \xbar) (\xi - \xbar)^T \text{,}
\end{align}
where we have that for $i=0$ $\Sigmaxx = \mat{0}$ and for $i >0$ :
\begin{align}
	\xi - \xbar & = \m \pm   \mathcol{\sqrt{(\Q + \k) \C}}{i} - \m \\
 	(\xi - \xbar) (\xi - \xbar)^T & = (\Q + \k)  \mathcol{\sqrt{\C}}{i}  \mathcol{\sqrt{\C}}{i}^T \text{,}
\end{align}
therefore:
\begin{align}
	\Sigmaxx & = 2 \sum_{i=1}^{\Q}  \frac{(\Q + \k)}{2 (\Q + \k)} \mathcol{\sqrt{\C}}{i}  \mathcol{\sqrt{\C}}{i}^T\\
	&= (\sqrt{\C})^T  \sqrt{\C} \\
	& = \C \text{.}
\end{align}
Therefore, Equations \eqref{eq:b-w} and  \eqref{eq:A-w} becomes:
\begin{align}
	\b &= \ybar - \A \m \text{,} \\
	\A &= \Sigmayx \C^{-1} 
	\text{,}
\end{align}
where $\ybar$ is given in Equation \eqref{eq:ybar-ut}; and 
$\Sigmayx$ is given in Equation  \eqref{eq:Sigmayx-ut}. 

\todo{figure out if cholesky should be lower or upper triangular}
%
\section{Multi-purpose Unscented Kitchen Sinks}
Would not we like a multi-purpose unscented kitchen sink :-) ? In a Random Kitchen Sink (\rks)
approximation to the multiple output version of the unscented Gaussian process (\ugp, \fix{Steinberg
and Bonilla, 2014}) we need to linearize: $g(\fn)$, where $\fn$ is a $\Q$-dimensional random variable corresponding to the $\Q$ latent function values at datapoint $n$:
\begin{align}
	g(\fn) & \approx \An \fn + \bn \quad \text{with }\\
	\label{eq:fn}
\fn &=  \W \phin \text{.}
\end{align}
The main point to notice here is that we are interested in linearizing $g(\fn)$ as a function of $\fn$. 
Therefore, we will work on $\fn$-space and use Equation \eqref{eq:fn} to compute the corresponding
moments. For the mean we have:
\begin{equation}
	\mufn = \M \phin \text{,}
\end{equation}
where $M$ is a $\Q \times \D$  matrix with the posterior means $\{ \m_q\}_{q=1}^{Q}$ (over 
the weights) on its rows. \\

Interestingly, the covariance $\Sigmafn$ has a diagonal structure as, by definition of our 
approximate posterior, there is no correlation across latent functions, i.e.:
\begin{equation}
	\fnq = \sum_{d=1}^{\D} \phi_{nd} w_{dq} \text{.}
\end{equation}
Therefore:
\begin{align}
	\Sigmafn &= \diag(\var{\fn}) \text{,} \quad \text{where } \\
	\label{eq:var-fnq}
	\var{\fnq} & = (\phin \hada \phin)^T \diag(\C_q)  \text{,}
\end{align}
where $\var{\fnq}$ denotes the variance if $\fnq$. \\

With this, we can compute the corresponding sigma points in $\fn$-space  as a function of 
$\M$ and $\{ \C_q \}$:
\begin{align}
	\xo & = \mufn = \M \phin \\
	\xi & = \M \phin  + \mathcol{\sqrt{ (\Q + \k) \Sigmafn}}{i} \quad i=1, \ldots, \Q \\
	\xi & = \M \phin  - \mathcol{\sqrt{ (\Q + \k) \Sigmafn}}{i-Q} \quad i=Q+1, \ldots, 2\Q 	\text{.}
\end{align}
The expressions above can be further simplified by considering:
\begin{align}
	\mathcol{\sqrt{ (\Q + \k) \Sigmafn}}{i} &= \sqrt{ (\Q + \k)} \mathcol{\sqrt{\diag (\var{\fn})} }{i} \\
	&=  \sqrt{ (\Q + \k)}  \sigmafn{i} \text{,} \quad \text{where }  \sigmafn{i} \in \Re^\Q \text{ and} \\
	\sigmafn{i} &= [0, 0, \ldots, \underbrace{\sqrt{\var{\fni}}}_{i\mth}, \ldots, 0 ]^T \text{,}
\end{align}
and $\var{\fni}$ is defined as in Equation \eqref{eq:var-fnq}. Therefore, the sigma-points can 
be rewritten as:
\begin{align}
\xo & = \mufn = \M \phin \\
	\xi & = \M \phin +  \sqrt{ (\Q + \k)}  \sigmafn{i} \quad i=1, \ldots, \Q \\
	\xi & = \M \phin - \sqrt{ (\Q + \k)}  \sigmafn{i-Q} \quad i=Q+1, \ldots, 2\Q \text{,}
\end{align}
 and the corresponding observations, weights and stats as before:
 \begin{align}
	\yi & = g(\xi) \quad i = 0, \quad \ldots 2\Q\\
	\wo &= \frac{\k}{\Q + \k} \\
	\wi &= \frac{1}{2(\Q + \k)}  \quad  i = 1, \ldots, 2\Q  \\
	\ybar  &=  \sum_{i=0}^{2 \Q} \wi \yi 	\\
	\Sigmayx  &= \sum_{i=0}^{2\Q} \wi (\yi - \ybar) (\xi - \M \phin)^T \\
	\Sigmaxx &= \Sigmafn^{-1} \text{.}
\end{align}
With this, the optimal linearization parameters are:
\begin{align}
	\bn & = \ybar - \An \M \phin \\
	\An &= \Sigmayx \Sigmafn^{-1} \text{,}
\end{align}
where we emphasize that  $\Sigmafn$ is a diagonal matrix.




\end{document}

































